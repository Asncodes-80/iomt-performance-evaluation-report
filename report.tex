\documentclass[a4paper]{article}
\usepackage{forest}
\usepackage{float}
\usepackage{geometry}
\usepackage{listings}
\usepackage{hyperref}
\usepackage{graphicx}
\usepackage{ragged2e}
\usepackage{color}
\usepackage{xepersian}
\usepackage{subfiles}
\newgeometry{left=1.4cm, right=1.4cm, bottom=2.0cm, top=2.0cm}
\settextfont[Scale=1]{XB Roya}

\DeclareMathSizes{12}{30}{16}{12}
\newgeometry{left=1.4cm, right=1.4cm, bottom=2.0cm, top=2.0cm}
\settextfont[Scale=1]{XB Roya}
\renewcommand{\baselinestretch}{1.5}

\newcommand{\equate}[1]{
    \begin{fleqn}
        \begin{gather}
            #1
        \end{gather}
    \end{fleqn}
}

\title{گزارش ارزیابی کارایی سیستم‌های اینترنت اشیا پزشکی و اینترنت اشیا براساس
مجموعه داده‌های \lr{CICIoMT2024} و مدلینگ ریاضیاتی \\ استاد ناظر: آقای دکتر مهدی
امینیان}
\author{علیرضا سلطانی نشان}

\begin{document}
\maketitle
\tableofcontents

\section*{مجوز}

به فایل license همراه این برگه توجه کنید. این برگه تحت مجوز GPLv3 منتشر شده است
که اجازه نشر و استفاده (کد و خروجی/pdf) را رایگان می‌دهد.

مهم‌ترین انگیزه برای توسعه این تحقیق وجود کمبود در داده‌های موجود ارزیابی کارایی
تجهیزات اینترنت اشیا پزشکی و پیشرفت امنیتی تمام شبکه‌هایی که در خصوص جریان‌های
داده‌ای و پردازش داده‌های پزشکی کار می‌کنند، می‌باشد بخصوص برای دستگاه‌های
اینترنت اشیا پزشکی به دلیل اطلاعات حیاتی‌ای که می‌توان به واسطه آن‌ها از بیماران
با بیماری‌های مختلف مانیتور و دریافت کرد.

با توسعه یک مجموعه داده بنچمارکینگ جدید و قابل تحقیق 

\newpage
\bibliographystyle{unsrt-fa}
\bibliography{refs.bib}
\end{document}